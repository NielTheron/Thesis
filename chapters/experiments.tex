\mychapter{Experiments}{Experiments}{}
\label{chap:experiment}

%========================================================================================================================================================================
\mysection{Introduction}{Introduction}
\label{sec:expintro}

This chapter aims to validate the theoretical concepts discussed in the previous chapters through the use of the developed simulation environment. This chapter will begin with describing the test setup in the simulation environment, afterwhich three tests will be run. The first test will determine the accuracy of the sensor itself and the performance of the state estimator in sensor fusion with other sensors. Then a test is run where the systems ability to accurately determine the pose of the satellite depending of the amount of features present in the image. The last test investigates the robustness of the system against temporal effects.

%==================================================================================================================================================================
\mysection{Test Configuration}{Test Configuration}
\label{sec:BaseTest}

The simulation environment is developed in MATLAB. Each test is conducted using a high-inclination orbit with a 2-minute pass over Cape Town, spanning from Cape Town Stadium to Port Alfred in the Eastern Cape, as illustrated in Figure \ref{fig:SatImg}. The initial conditions are specified in Table \ref{tab:SimulationParameters} and serve as the default configuration for all tests. Any deviations from these default values are explicitly stated in the respective test descriptions.

\begin{figure}[H]
    \centering
    % Row 1
    \begin{subfigure}{0.45\textwidth}
        \includegraphics[width=\linewidth]{figures/experiments/sat_image_001.png}
        \caption{Satellite image taken at t = 0s}
    \end{subfigure}
    \hfill
    \begin{subfigure}{0.45\textwidth}
        \includegraphics[width=\linewidth]{figures/experiments/sat_image_901.png}
        \caption{Satellite image taken at t = 30s}
    \end{subfigure}
    
    % Row 2
    \vspace{0.5cm}
    \begin{subfigure}{0.45\textwidth}
        \includegraphics[width=\linewidth]{figures/experiments/sat_image_1801.png}
        \caption{Satellite image taken at t = 60s}
    \end{subfigure}
    \hfill
    \begin{subfigure}{0.45\textwidth}
        \includegraphics[width=\linewidth]{figures/experiments/sat_image_3571.png}
        \caption{Satellite image taken at t = 120s}
    \end{subfigure}
    
    \caption{Satellite images captured at different time steps of the simulation. Cape Town begin shown in the first image and Port Alfred in the last.}
    \label{fig:SatImg}
\end{figure}

\begin{table}[H]
    \centering
    \begin{tabular}{|l|l|l|l|}
        \hline
        \multicolumn{4}{|c|}{Constants} \\ \hline
        Attribute        & Simulator                          & Filter                               & Units \\ \hline
        Sample Rate      & $60$                               & $60$                                 & Hz \\
        $GM$             & $3.986 \times 10^5$                & $3.986 \times 10^5$                  & km$^3$/s$^2$ \\
        $R_{earth}$      & $6.378 \times 10^3$                & $6.378 \times 10^3$                  & km \\
        $J_2$            & $1.082 \times 10^{-3}$             & $0$                                  & -- \\
        $\omega_e$       & $7.292 \times 10^{-5}$             & $7.292 \times 10^{-5}$               & rad/s \\
        $\mathbf{I}$     & $\text{diag}(1,1,1)$               & $\text{diag}(1.05,0.95,1.02)$        & kg·m$^{2}$ \\ \hline
        \multicolumn{4}{|c|}{Initial States} \\ \hline
        Attribute        & Simulator           & Filter            & Units \\ \hline
        Latitude         & $-33.90$            & $-33.90$          & $^\circ$ \\
        Longitude        & $18.41$             & $18.41$           & $^\circ$ \\
        Altitude         & $500$               & $500$             & km \\
        Roll             & $0$                 & $0$               & $^\circ$ \\
        Pitch            & $0$                 & $0$               & $^\circ$ \\
        Yaw              & $0$                 & $0$               & $^\circ$ \\
        Roll Rate        & $2$                 & $2$               & $^\circ$/s \\
        Pitch Rate       & $0$                 & $0$               & $^\circ$/s \\
        Yaw Rate         & $0$                 & $0$               & $^\circ$/s \\ \hline
    \end{tabular}
    \caption{Comparison of the constants and initial state parameters used in the simulation environment and the EKF implementation.}
    \label{tab:SimulationParameters}
\end{table}


\noindent
The camera parameters used in the simulation are standardized as listed in Table \ref{tab:CameraParameters}. The camera model is derived from the TriScape100x satellite imager developed by Simera Sense, illustrated in Figure \ref{fig:TSC} \cite{TriScape}. It retains the same focal length as the original sensor but uses a reduced image resolution to improve processing efficiency within the simulation environment. Furthermore, the pixel size has been increased to match the minimum ground sampling distance (GSD) of 15 m, consistent with the Sentinel-2 imagery obtained from the Copernicus data browser.

\begin{table}[H]
    \centering
    \begin{tabular}{|l|l|l|l|}
        \hline
        Characteristic          & Simulated Camera & Simera Sense Camera & Units \\ \hline
        Horizontal Resolution   & 720   & 4096      & pixel \\
        Vertical Resolution     & 720   & 3072      & pixel \\
        Focal Length            & 580   & 580       & mm \\
        Pitch                   & 17.4  & 5.5       & $\mu$m \\
        GSD @ 500 km            & 15    & 4.75      & m/pixel \\
        Swath                   & 10.8  & 19.4      & km \\ \hline
    \end{tabular}
    \caption{Comparison of the camera characteristics used in the simulation with those of the real Simera Sense TriScape camera.}
    \label{tab:CameraParameters}
\end{table}


\begin{figure}[H]
    \centering
    \includegraphics[width=0.5\textwidth]{figures/experiments/TriScapeCamera.png}
    \caption{The Simera Sense TriScape camera used as the primary imaging payload for Earth observation in the simulation study. \cite{TriScape}}
    \label{fig:TSC}
\end{figure}

\noindent
The Extended Kalman Filter (EKF) is initialised with appropriate uncertainty parameters to reflect the initial state estimate accuracy and system dynamics. The covariance matrix $\mathbf{P}_0$ represents the initial uncertainty in the state estimate, with diagonal elements corresponding to the variance of position, velocity, attitude and angular velocity. The process noise covariance matrix $\mathbf{Q}$ characterizes the uncertainty introduced by unmodeled dynamics and disturbances during state propagation. These matrices are initialised using the parameters specified below in Table \ref{tab:CovarianceParameters}, which have been tuned to balance filter responsiveness and stability throughout the simulation.

\begin{table}[H]
\centering
\begin{tabular}{|c|c|c|}
\hline
\textbf{State} & $\text{diag}(\mathbf{P}_0)$ & $\text{diag}(\mathbf{Q})$ \\ \hline
Position $[r_x, r_y, r_z]$ & $[500, 500, 500]^2$ m$^2$ & $[100, 100, 100]^2$ m$^2$ \\
Velocity $[v_x, v_y, v_z]$ & $[10, 10, 10]^2$ (m/s)$^2$ & $[2, 2, 2]^2$ (m/s)$^2$ \\
Attitude (Att) & $(10^\circ)^2$ & $(5^\circ)^2$ \\
Angular velocity $[w_x, w_y, w_z]$ & $[1, 1, 1]^2$ ($^\circ$/s)$^2$ & $[0.5, 0.5, 0.5]^2$ ($^\circ$/s)$^2$ \\ \hline
\end{tabular}
\caption{Diagonal entries of the initial covariance ($\mathbf{P}_0$) and process noise ($\mathbf{Q}$) matrices. The listed values are standard deviations ($\sigma$) and are squared to form the actual covariance values.}
\label{tab:CovarianceParameters}
\end{table}


\noindent
Figure \ref{fig:FilterError} illustrates the discrepancies between the simulator and the filter predictions. This scenario corresponds to the filter estimating the system states without incorporating any sensor measurements, relying entirely on the model dynamics. The model dynamics include the $J_2$ pertubations and a small rotational offset of 144 arcseconds in the quaternion model. The Figure \ref{fig:FilterError} highlights how the state estimates deviate over time due to model uncertainties and unmodeled dynamics, which is expected.

\begin{figure}[H]
    \centering
    \includegraphics[width=\linewidth]{figures/experiments/FilterError.pdf}
    \caption{Position and attitude estimation error of the filter with no measurement inputs.}
    \label{fig:FilterError}
\end{figure}

%========================================================================================================================================================================
\mysection{Sensor Testing}{Sensor Testing}
\label{sec:SensorTest}

\noindent
To assess the performance and contribution of the Earth Tracker within the satellite pose estimation framework, a series of systematic sensor tests are conducted. These experiments isolate and compare various sensor configurations to evaluate accuracy, robustness, and the complementary nature of vision-based and traditional navigation sensors. The testing begins with a rigorous evaluation of the Earth Tracker in standalone mode, followed by four integrated configurations: the Star Tracker standalone, the traditional ADCS suite (comprising the GPS, gyroscope, coarse sun sensor, and magnetometer), the Star Tracker integrated with the ADCS suite, and the Earth Tracker integrated with the ADCS suite. Through this comparative analysis, the study examines the Earth Tracker's relative accuracy, its contribution to enhancing ADCS performance, and its potential as an alternative to Star Trackers in resource-constrained satellite missions. The findings establish a foundation for evaluating the Earth Tracker's practical applicability in autonomous on-orbit attitude and position estimation.

%===============================================================================================================================================================
\mysubsection{Earth Tracker Standalone}{Earth Tracker Standalone}

\noindent
This test evaluates the Earth Tracker's standalone pose estimation capabilities without fusion from auxiliary sensors. Five configurations are tested to assess performance under different measurement uncertainty conditions, as specified in Table \ref{tab:ET_characteristics}.

\begin{table}[H]
\centering
\begin{tabular}{|c|c|c|}
\hline
\textbf{Configuration} & \textbf{Sample Rate} & \textbf{Noise ($\sigma$)} \\ \hline
ET Alone (High Trust)  &  1 Hz & 10 m \\ \hline
ET Alone (Low Trust) &  1 Hz & 1500 m \\ \hline
ET Attitude Only & 1 Hz & 10 m \\ \hline
ET Position Only & 1 Hz & 10 m \\ \hline
ET + GPS & 1 Hz & 10 m \\ \hline
\end{tabular}
\caption{Earth Tracker sensor configurations for standalone testing.}
\label{tab:ET_characteristics}
\end{table}

\noindent
Figures \ref{fig:ETHighTrust} and \ref{fig:ETLowTrust} illustrate that when the Earth Tracker is assigned a high trust level, the EKF relies almost entirely on its measurements to determine position and attitude. Under these conditions, the position error reaches approximately 1.53 km, while the attitude error remains at 0.17$^{\circ}$. In the low-trust configuration, clear discrepancies appear between the simulated and estimated states, as previously shown in Figure \ref{fig:FilterError}. Although the attitude estimation initially demonstrates high accuracy, it gradually begins to diverge over time, due to the increasing position error which effects the Earth Tracker measurements.


\begin{figure}[H]
    \centering
    \includegraphics[width=\linewidth]{figures/experiments/ETHighTrust.pdf}
    \caption{High trust configuration for position and attitude estimation of the Earth Tracker.}
    \label{fig:ETHighTrust}
\end{figure}

\begin{figure}[H]
    \centering
    \includegraphics[width=\linewidth]{figures/experiments/ETLowTrust.pdf}
    \caption{Low trust configuration for position and attitude estimation of the Earth Tracker.}
    \label{fig:ETLowTrust}
\end{figure}

\noindent
Figures \ref{fig:ETAttitudeOnly} to \ref{fig:ETGPS} demonstrate that when only attitude is estimated, the attitude error decreases substantially from 587.86 arcseconds to 8.697 arcseconds. This result suggests a coupling between position and attitude estimation within the measurement model, where uncertainty in position propagates into the attitude estimate. Interestingly, Figure \ref{fig:ETPositionOnly} shows that estimating position alone does not improve position accuracy, implying that accurate position knowledge plays a more critical role in enhancing attitude estimation than the reverse. The position-only estimation is also highly sensitive to the 500 km lens characteristic assumption used in the Earth Tracker. Further research is needed to develop a dynamic altitude sensing approach, potentially through pattern or batch feature processing rather than estimating with one feature at a time. As shown in Figure \ref{fig:ETGPS}, providing the satellite's position information with the GPS significantly improves attitude accuracy to 18.45 arcseconds, indicating that this configuration presents a viable and effective solution.

\begin{figure}[H]
    \centering
    \includegraphics[width=\linewidth]{figures/experiments/ETAttitudeOnly.pdf}
    \caption{Position and attitude error of Earth Tracker in attitude only configureration.}
    \label{fig:ETAttitudeOnly}
\end{figure}


\begin{figure}[H]
    \centering
    \includegraphics[width=\linewidth]{figures/experiments/ETPositionOnly.pdf}
    \caption{Position and attitude error of Earth Tracker in positioin only configuration.}
    \label{fig:ETPositionOnly}
\end{figure}

\begin{figure}[H]
    \centering
    \includegraphics[width=\linewidth]{figures/experiments/ETGPS.pdf}
    \caption{Position and attitude estimation error of Earth Tracker and GPS configuration.}
    \label{fig:ETGPS}
\end{figure}

\noindent
Table \ref{tab:ET_errors} further clarifies the performance trends observed across the different Earth Tracker configurations. The high-trust standalone Earth Tracker exhibits the poorest overall performance, with an average attitude error of 587 arcseconds, primarily due to the filter's overreliance on potentially noisy visual measurements and the Earth Trackers 2D-3D conversion model. In contrast, when the Earth Tracker is supplemented with accurate position information, either through GPS or other aiding sensors, the attitude estimation improves dramatically to 18.45 arcseconds. This result highlights the strong dependency between accurate positional knowledge and attitude determination in the Earth Tracker's measurement model. It also reinforces the potential of fusing visual and positional data to achieve reliable six-degree-of-freedom state estimation, particularly in scenarios where a star tracker is unavailable or infeasible.

\begin{table}[H]
\centering
\begin{tabular}{|c|c|c|}
\hline
\textbf{Configuration} & \textbf{Average Pos Error (m)} & \textbf{Average Att Error (arcseconds)} \\ \hline
ET (High Trust)     & 1530  & 587.8 \\ \hline
ET (Low Trust)      & 72    & 52.42 \\ \hline
ET Attitude Only    & -     & 8.697 \\ \hline
ET Position Only    & 1522  & -     \\ \hline
ET + GPS            & 10    & 18.45 \\ \hline 
\end{tabular}
\caption{Comparison of position and attitude errors across different ET sensor configurations.}
\label{tab:ET_errors}
\end{table}


%===============================================================================================================================================================
\mysubsection{Earth Tracker and Star Tracker in an ADCS System}{Earth Tracker and Star Tracker in an ADCS System}

\noindent
This experiment compares the performance of the Earth Tracker and the Star Tracker within an ADCS. The analysis begins by evaluating the star tracker independently to assess it's capabilities, followed by the Earth Tracker and Star Tracker integration within the full ADCS sensor suite. The ADCS suite comprises a GPS receiver, gyroscope, magnetometer, and coarse sun sensor, serving as a baseline configuration for evaluating the performance enhancements provided by the Earth and Star Trackers.
\vspace{0.5cm}

\noindent
The order in which sensor measurements are processed during the sequential Extended Kalman Filter (EKF) update cycle is crucial for achieving optimal estimation performance. As summarized in Table \ref{tab:Rank}, sensors that provide fundamental state information are prioritized early in the update sequence, while progressively refined measurements are applied in later stages. This ordering minimizes the propagation of uncertainty through the estimation process.

\begin{table}[H]
\centering
\begin{tabular}{|c|c|p{9cm}|}
\hline
\textbf{Rank} & \textbf{Sensor} & \textbf{Justification} \\ \hline
1 & GPS & Provides direct position measurements, establishing the translational state and aiding subsequent attitude estimation. \\ \hline
2 & Gyro & Supplies angular velocity measurements, refining rotational dynamics before attitude sensor updates. \\ \hline
3 & CSS & Coarsest attitude sensor; provides sun vector measurements with moderate accuracy. \\ \hline
4 & MAG & More accurate than CSS; provides magnetic field vector measurements for attitude determination. \\ \hline
5 & ET & Provides full six-degree-of-freedom state estimation, combining visual data for both position and attitude determination. \\ \hline
6 & ST & Most accurate attitude sensor; provides final refinement to the state estimate. \\ \hline
\end{tabular}
\caption{Ranking of sensors based on performance and suitability in the sequential EKF update process.}
\label{tab:Rank}
\end{table}

\noindent
The ADCS suite sensors are configured according to the parameters listed in Table \ref{tab:ADCS_characteristics}, representing typical performance specifications for CubeSat-class hardware.

\begin{table}[H]
\centering
\begin{tabular}{|c|c|c|c|}
\hline
\textbf{Sensor} & \textbf{Sample Rate} & \textbf{Noise ($\sigma$)} & \textbf{Drift} \\ \hline
ET              &  1 Hz     & 10 m                  & -                     \\ \hline
ST              &  1 Hz     & 30 arcseconds         & -                     \\ \hline
GPS             &  10 Hz    & 5 m                   & 0.1 m/$\sqrt{hr}$     \\ \hline
Gyro            &  30 Hz    & 0.01 $^{\circ}$/s     & 0.001 $^{\circ}$/$\sqrt{hr}$   \\ \hline
Magnetometer    &  10 Hz    & 500 nT                & -                     \\ \hline
CSS             &  10 Hz    & 5\%                   & -                     \\ \hline
\end{tabular}
\caption{ADCS sensor configurations used for standalone and combined testing, including sampling rate, measurement noise, and drift parameters.}
\label{tab:ADCS_characteristics}
\end{table}

\noindent
Figure \ref{fig:STOnly} illustrates that the Star Tracker alone can estimate attitude with high accuracy, achieving an error of approximately 39.6 arcseconds. However, it cannot estimate position, leading to position drift due to model discrepancies in the filter. Conversely, Figure \ref{fig:ADCSOnly} shows that the ADCS suite, when operated without either the Star Tracker or Earth Tracker, maintains good position estimation but produces a relatively coarse attitude estimate of 3.37$^{\circ}$.

\begin{figure}[H]
    \centering
    \includegraphics[width=\linewidth]{figures/experiments/STOnly.pdf}
    \caption{Star Tracker position and attitude estimation error in the EKF.}
    \label{fig:STOnly}
\end{figure}

\begin{figure}[H]
    \centering
    \includegraphics[width=\linewidth]{figures/experiments/ADCSOnly.pdf}
    \caption{ADCS Suite position and attitude estimation error in the EKF.}
    \label{fig:ADCSOnly}
\end{figure}

\noindent
Figures \ref{fig:STADCS} and \ref{fig:ETADCS} directly compare the performance of the Earth Tracker and Star Tracker when integrated with the ADCS suite. The Star Tracker maintains a slight performance advantage, achieving an average attitude error of 1.04$^{\circ}$ compared to 1.26$^{\circ}$ for the Earth Tracker. In both configurations, position estimation is primarily supported by the GPS measurements.

\begin{figure}[H]
    \centering
    \includegraphics[width=\linewidth]{figures/experiments/STADCS.pdf}
    \caption{Position and attitude estimation errors for the Star Tracker integrated with the ADCS suite.}
    \label{fig:STADCS}
\end{figure}

\begin{figure}[H]
    \centering
    \includegraphics[width=\linewidth]{figures/experiments/ETADCS.pdf}
    \caption{Position and attitude estimation errors for the Earth Tracker integrated with the ADCS suite.}
    \label{fig:ETADCS}
\end{figure}

\noindent
As summarized in Table \ref{tab:ADCS_Stats}, both trackers enhance the ADCS suite's overall estimation performance by roughly 3 times. While the Star Tracker provides marginally better accuracy, the Earth Tracker demonstrates significant promise as a practical alternative, particularly considering it can utilize the existing payload camera onboard a satellite without requiring additional dedicated hardware. It is worth noting in Figure \ref{fig:FeatureDetection} that the Earth Tracker achieved an accuracy of 8.69 arcseconds. This indicates that the ADCS suite did not reach this level of precision, as the CSS and magnetometer are sampled at 10 Hz, while the Earth Tracker and Star Tracker are sampled only at 1 Hz. This suggests that more in-depth tuning of the EKF is required to achieve better performance. Nevertheless, the Earth Tracker demonstrates a significant improvement, comparable to that of the Star Tracker, even within a roughly tuned filter.
 

\begin{table}[H]
\centering
\begin{tabular}{|c|c|c|}
\hline
\textbf{Configuration} & \textbf{Average Position Error (m)} & \textbf{Average Attitude Error ($^{\circ}$)} \\ \hline
ST Alone        & 11     & 39.58 (arcseconds) \\ \hline
ADCS Alone      & 8      & 3.37     \\ \hline
ST + ADCS       & 8      & 1.04     \\ \hline
ET + ADCS       & 11     & 1.26     \\ \hline
\end{tabular}
\caption{Comparison of position and attitude estimation accuracy across sensor configurations.}
\label{tab:ADCS_Stats}
\end{table}

%===============================================================================================================================================================
\mysubsection{Summary of Results}{Summary of Results}

\noindent
Table \ref{tab:experiment_errors} summarizes the estimation performance across all tested configurations. The results highlight key trade-offs between measurement reliability, sensor fusion strategies, and achievable accuracy. Notably, the Earth Tracker  attitude only estimation accuracy of 8.69 arcseconds, demonstrating that even under simplified simulation conditions, the Earth Tracker can meaningfully enhance spacecraft attitude determination.

\begin{table}[H]
\centering
\begin{tabular}{|c|c|c|}
\hline
\textbf{Configuration} & \textbf{Average Pos Error (m)} & \textbf{Average Att Error ($^{\circ}$)} \\ \hline
ET Alone (High Trust)   & 1530 & 587.8 arcseconds               \\ \hline
ET Alone (Low Trust)    & 71   & 52.42 arcseconds               \\ \hline
ET Attitude Only        & -    & 8.697 arcseconds               \\ \hline
ET Position Only        & 1522 & -                              \\ \hline
ET + GPS                & 10   & 18.45 arcseconds               \\ \hline
ST Alone                & 11   & 39.58 arcseconds               \\ \hline
ADCS Suite Standalone   & 8    & 3.37                           \\ \hline
ET + ADCS Suite         & 11   & 1.26                           \\ \hline
ST + ADCS Suite         & 8    & 1.04                           \\ \hline
\end{tabular}
\caption{Summary of position and attitude estimation errors across all tested configurations.}
\label{tab:experiment_errors}
\end{table}

%=======================================================================================================

\mysection{Feature Testing}{Feature Testing}

\noindent
Feature testing evaluates the sensitivity and reliability of the Earth Tracker’s vision-based measurement model under varying image-processing conditions. The experiments aim to quantify the influence of feature detection algorithms, feature count, and lens distortion on pose estimation accuracy. Specifically, the performance of SIFT, SURF, and ORB detectors is assessed across multiple feature configurations, followed by an analysis of optical distortion effects on feature geometry and matching consistency. These investigations establish the practical limitations of the Earth Tracker’s visual subsystem and inform the selection of suitable detection and calibration strategies for on-board implementation these test are done with the same setup as Table \ref{tab:CovarianceParameters} and the high trust ET model.

%===========================================================================================================================================================
\mysubsection{Feature Detector Testing}{Feature Detector Testing}

\noindent
Figure \ref{fig:ETMeasurements} illustrates the comparison between the true and estimated Earth Tracker measurements for Feature 1 in the body reference frame, demonstrating close alignment between the two datasets. The measurement error in Figure \ref{fig:ETMeasurementError} confirms that the difference between the true and estimated feature vectors remains below 200 meters, indicating high accuracy in the feature detection and back-projection process. These errors can be caused by the differences in the WGS ellipsoid model and the 500 km lens characteristic assumption made for the 2D-3D back projection process. 

\begin{figure}[H]
    \centering
    \includegraphics[width=0.9\linewidth]{figures/experiments/ETMeasurements.pdf}
    \caption{Comparison between the true and estimated Earth Tracker measurements for Feature 1.}
    \label{fig:ETMeasurements}
\end{figure}

\begin{figure}[H]
    \centering
    \includegraphics[width=0.9\linewidth]{figures/experiments/ETMeasurementError.pdf}
    \caption{Measurement error between the true and estimated Earth Tracker feature vector for Feature 1.}
    \label{fig:ETMeasurementError}
\end{figure}

\noindent
To evaluate the suitability of different feature detection methods for the Earth Tracker system, three commonly used algorithms SIFT, SURF, and ORB are tested. Each detector is evaluated using configurations of 1, 5, and 10 detected features to investigate how the number of features influences position and attitude estimation accuracy.

\begin{table}[H]
\centering
\begin{tabular}{|c|c|c|c|c|c|c|}
\hline
\textbf{Detector} & \multicolumn{2}{c|}{\textbf{1 Feature}} & \multicolumn{2}{c|}{\textbf{5 Features}} & \multicolumn{2}{c|}{\textbf{10 Features}} \\ \hline
 & \textbf{Pos (m)} & \textbf{Att ($^{\circ}$)} & \textbf{Pos (m)} & \textbf{Att ($^{\circ}$)} & \textbf{Pos (m)} & \textbf{Att ($^{\circ}$)} \\ \hline
SIFT & 464 & 0.554  & 1489 & 0.160 & 1516 & 0.162 \\ \hline
SURF & 465 & 0.525 & 1509 & 0.162 & 1527 & 0.163 \\ \hline
ORB  & 495 & 0.572 & 1452 & 0.157 & 1495 & 0.160 \\ \hline
\end{tabular}
\caption{Feature detector performance comparison showing average position and attitude errors.}
\label{tab:feature_detectors_split}
\end{table}

\begin{figure}[H]
    \centering
    
    % Row 1
    \begin{subfigure}{0.3\textwidth}
        \includegraphics[width=\linewidth]{figures/experiments/ESTSIFT1.png}
        \caption{SIFT detection with 1 feature at $t = 0.$}
    \end{subfigure}
    \hfill
    \begin{subfigure}{0.3\textwidth}
        \includegraphics[width=\linewidth]{figures/experiments/ESTSURF1.png}
        \caption{SURF detection with 1 feature at $t = 0.$}
    \end{subfigure}
    \hfill
    \begin{subfigure}{0.3\textwidth}
        \includegraphics[width=\linewidth]{figures/experiments/ESTORB1.png}
        \caption{ORB detection with 1 feature at $t = 0.$}
    \end{subfigure}
    
    % Row 2
    \vspace{0.5cm}
    \begin{subfigure}{0.3\textwidth}
        \includegraphics[width=\linewidth]{figures/experiments/ESTSIFT10.png}
        \caption{SIFT detection with 10 features at $t = 0.$}
    \end{subfigure}
    \hfill
    \begin{subfigure}{0.3\textwidth}
        \includegraphics[width=\linewidth]{figures/experiments/ESTSURF10.png}
        \caption{SURF detection with 10 features at $t = 0.$}
    \end{subfigure}
    \hfill
    \begin{subfigure}{0.3\textwidth}
        \includegraphics[width=\linewidth]{figures/experiments/ESTORB10.png}
        \caption{ORB detection with 10 features at $t = 0.$}
    \end{subfigure}
    
    \caption{Comparison of feature detector outputs for 1 and 10 detected features.}
    \label{fig:3x2}
\end{figure}

\noindent
The results indicate that using a single detected feature provides insufficient geometric constraints for accurate pose estimation. With only one feature, all detectors exhibit attitude errors exceeding 0.5$^{\circ}$, while position accuracy remains moderate (approximately 470-500 m). This confirms that multiple spatially diverse features are necessary for robust orientation estimation.

\vspace{0.5cm}
\noindent
As the number of detected features increases, attitude accuracy improves significantly. From Table \ref{tab:feature_detectors_split} and Figure \ref{fig:3x2}, configurations using five or more features achieve high accuracy levels, with attitude errors ranging between 0.157$^{\circ}$ and 0.162$^{\circ}$. This improvement highlights the importance of feature diversity in providing sufficient geometric information for accurate attitude determination. However, the mean position errors unexpectedly increase with additional features,from approximately 470 m with one feature to around 1500 m with ten features. Interestingly as little as five features where needed to obtain the saturation point of the Earth Tracker capabilties, indicating that even with few robust features the Earth Tracker can be a viable sensor. 

\vspace{0.5cm}
\noindent
This counterintuitive trend arises from the inherent altitude ambiguity in monocular vision systems. As the number of features increases, small systematic biases in back-projected depth accumulate, introducing correlated errors in position estimates. This effect emphasizes a key limitation of single-camera Earth tracking systems: without auxiliary altitude or range data, position estimation remains sensitive to accumulated bias in 2D-to-3D reconstruction. Future implementations could mitigate this limitation by incorporating auxiliary altitude data or bias-compensation techniques within the Kalman Filter.

\vspace{0.5cm}
\noindent
Comparing the detectors, SURF achieves the best overall accuracy, particularly in attitude estimation, with the lowest error of 0.525$^{\circ}$ for single-feature tests and consistent performance across multi-feature configurations. ORB achieves the lowest position errors for 5 and 10 features (1452 m and 1495 m, respectively) while maintaining competitive attitude accuracy (0.157$^{\circ}$-0.572$^{\circ}$). SIFT performs similarly to SURF, though it exhibits slightly higher attitude errors for the single-feature case. The performance differences among the three detectors remain within 15\% for both position and attitude metrics, suggesting that selection should primarily consider computational efficiency. Notably, even on on the high-perfomance desktopo hardware used in this simulation, ORB executed significantly faster than SIFT and SURF. For real-time implementation on resource constrained satellite processors with limited computational power and memory, ORB emerges as the most practical choice, offering competitive accuracy. 

%===============================================================================================================================================
\mysubsection{Lens Distortion Testing}{Lens Distortion Testing}

\noindent
Lens distortion effects are evaluated to assess their impact on Earth Tracker measurement accuracy. Three types of distortion are modeled: radial distortion (barrel and pincushion effects), tangential distortion (due to lens misalignment), and chromatic aberration (wavelength-dependent focusing).

\begin{table}[H]
\centering

\begin{tabular}{|c|c|c|}
\hline
\textbf{Distortion Type} & \textbf{Parameter} & \textbf{Value} \\ \hline
\multirow{3}{*}{Radial} & $k_1$ & 0.001 \\ 
 & $k_2$ & 0.001 \\ 
 & $k_3$ & 0.001 \\ \hline
\multirow{2}{*}{Tangential} & $p_1$ & 0.0001 \\ 
 & $p_2$ & 0.0001 \\ \hline
\multirow{3}{*}{Chromatic} & $R$ & 0.98 \\ 
 & $G$ & 1.00 \\ 
 & $B$ & 1.02 \\ \hline
\end{tabular}
\caption{Lens distortion parameters used in testing.}
\label{tab:lens_distortion}
\end{table}

\begin{figure}[H]
    \centering
    \begin{subfigure}[b]{0.48\linewidth}
        \centering
        \includegraphics[width=\linewidth]{figures/experiments/EILD2.png}
        \caption{Feature matching without lens distortion.}
        \label{fig:NoDistortion}
    \end{subfigure}
    \hfill
    \begin{subfigure}[b]{0.48\linewidth}
        \centering
        \includegraphics[width=\linewidth]{figures/experiments/EILD1.png}
        \caption{Feature matching with lens distortion.}
        \label{fig:WithDistortion}
    \end{subfigure}
    \caption{Comparison of Earth Tracker measurements with and without lens distortion correction.}
    \label{fig:LensDistortion}
\end{figure}

\noindent
The results reveal that even moderate lens distortion parameters, representative of typical commercial off-the-shelf camera systems, severely degrade Earth Tracker performance. As shown in Figure \ref{fig:LensDistortion}, uncorrected distortion introduces systematic pixel displacement errors that propagate through the back-projection process, resulting in incorrect 3D feature vector generation. The accumulated measurement errors render the Earth Tracker measurements effectively unusable for pose estimation, causing filter divergence and large state estimate errors.
\vspace{0.5cm}

\noindent
This sensitivity to lens distortion underscores the critical importance of accurate camera calibration and distortion correction as a preprocessing step. For operational implementations, the camera intrinsic parameters and distortion coefficients must be precisely characterized either through ground-based calibration using standard patterns (such as checkerboard targets) or through in-flight self-calibration techniques. Without proper distortion correction, the geometric accuracy assumptions underlying the Earth Tracker measurement model are violated, leading to significant performance degradation. Future work should prioritize robust calibration procedures and potentially explore distortion-invariant feature detection methods to enhance system resilience to optical imperfections.

%========================================================================================================================================================================
\mysection{Temporal Test}{Temporal Test}
\label{sec:TemporalTest}

\noindent
A critical practical consideration for vision-based navigation is the accurate timestamping of image measurements. Processing delays, transmission latencies, or synchronization errors can cause measurements to be associated with incorrect state estimates, introducing systematic errors into the filter. This test evaluates the Earth Tracker's robustness to timestamp errors by artificially introducing measurement delays and assessing their impact on pose estimation accuracy.
\vspace{0.5cm}

\noindent
The Earth Tracker is configured as shown in Table \ref{tab:ET_config}, operating at 1 Hz while the remainder of the simulation runs at 60 Hz. Delays are introduced by shifting the measurement sequence backward in time, simulating scenarios where the image capture timestamp does not correspond to the current filter propagation time.

\begin{table}[H]
\centering
\begin{tabular}{|c|c|c|}
\hline
\textbf{Sensor} & \textbf{Sample Rate} & \textbf{Noise ($\sigma$)} \\ \hline
ET  &  1 Hz & 10 m \\ \hline
\end{tabular}
\caption{Earth Tracker configuration for temporal delay testing.}
\label{tab:ET_config}
\end{table}

\noindent
Table \ref{tab:Delay_errors} demonstrates that both position and attitude estimation errors increase significantly with measurement delay. This behavior is expected, as features detected at a previous satellite position are incorrectly associated with the current state estimate. Given the satellite's orbital velocity of approximately 7 km/s, each second of delay corresponds to a 7 km spatial displacement. A 1-second delay results in position errors increasing from 1.53 km to 7.99 km, while a 3-second delay produces errors exceeding 23 km. Attitude errors exhibit similar degradation, increasing from 0.167$^{\circ}$ to 3.398$^{\circ}$ over the same delay range.

\begin{table}[H]
\centering
\begin{tabular}{|c|c|c|}
\hline
\textbf{Delay (s)} & \textbf{Average Position Error (km)} & \textbf{Average Attitude Error ($^{\circ}$)} \\ \hline
0 & 1.530  & 0.167 \\ \hline
1 & 7.990  & 1.67 \\ \hline
2 & 15.55  & 2.54 \\ \hline
3 & 23.18  & 3.398 \\ \hline 
\end{tabular}
\caption{Impact of measurement delay on Earth Tracker estimation accuracy.}
\label{tab:Delay_errors}
\end{table}

\begin{figure}[H]
    \centering
    \includegraphics[width=\linewidth]{figures/experiments/ETDelay0.pdf}
    \caption{Position and attitude estimation of the Earth Tracker with a 0 second delay.}
    \label{fig:ET0}
\end{figure}

\begin{figure}[H]
    \centering
    \includegraphics[width=\linewidth]{figures/experiments/ETDeley3.pdf}
    \caption{Position and atitude estimation of the Earth Tracker with a 3 second delay.}
    \label{fig:ET3}
\end{figure}


\noindent
Figures \ref{fig:ET0} and \ref{fig:ET3} visually illustrates the dramatic increase in attitude error when a 3-second delay is introduced. The results underscore the critical importance of precise image timestamping and tight synchronization between the camera trigger and the navigation filter clock. For low-rate image sensors (1-2 Hz), even small timing errors can cascade into significant systematic biases throughout the state estimate. Operational implementations must prioritize hardware-level timestamp capture at the moment of image exposure, minimize processing latency, and potentially incorporate measurement delay compensation techniques within the EKF to maintain estimation accuracy.
%=========================================================================================================================================================================
\mysection{Conclusion}{Conclusion}

\noindent
This chapter presented a comprehensive evaluation of the developed satellite pose estimation algorithms using the simulation environment. The experiments validated the theoretical models and demonstrated the performance of the Earth Tracker, Star Tracker, and the ADCS sensor suite under various configurations.
\vspace{0.5cm}

\noindent
The standalone Earth Tracker exhibited strong attitude estimation capabilities when aided with accurate position information, achieving an attitude error as low as 8.69 arcseconds.. Without position aiding, the standalone Earth Tracker in high-trust mode showed significant position errors (1.53 km) and degraded attitude performance (587.8 arcseconds), highlighting the strong coupling between position and attitude in the visual measurement model. Nevertheless, this still indicates that the Earth Tracker can be used as a standalone attitude sensor.
\vspace{0.5cm}

\noindent
Integrating the Earth Tracker or Star Tracker with the ADCS suite significantly improved estimation accuracy almost by 3 times. The Earth Tracker and Star Tracker contributed nearly the same level of improvement to the system. Although the results are somewhat inaccurate, than as expected, they indicate that with proper EKF tuning, the Earth Tracker can serve as a practical alternative when additional attitude sensors are unavailable.
\vspace{0.5cm}

\noindent
Feature testing revealed that increasing the number of detected features improved attitude accuracy, with errors reducing from 0.554° for a single SIFT feature to 0.162° for ten features, while position errors increased due to monocular altitude ambiguity (approximately 470-1516 m across configurations). Among feature detectors, SURF provided the most consistent attitude accuracy, whereas ORB offered competitive performance with lower computational cost, making it suitable for real-time CubeSat applications.
\vspace{0.5cm}

\noindent
Lens distortion tests emphasized the critical need for precise camera calibration. Uncorrected distortion produced large errors in 3D feature back-projection, rendering Earth Tracker measurements unreliable for pose estimation. Temporal tests showed that even small measurement delays significantly degrade accuracy: a 1-second delay increased position error from 1.53 km to 7.99 km and attitude error from 0.167° to 1.67°, while a 3-second delay further escalated errors to 23.18 km and 3.398°, respectively.
\vspace{0.5cm}

\noindent
Overall, the experiments demonstrate that the Earth Tracker, when combined with auxiliary sensors such as GPS, can provide accurate six-degree-of-freedom state estimation for satellite operations. The results highlight the importance of multi-feature detection, careful calibration, and precise timing for vision-based navigation systems, providing clear guidance for practical implementation in resource-constrained satellite missions.