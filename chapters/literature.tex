\mychapter{Literature}{Literature}{}
\label{chap:literature}


\mysection{Introduction}{Introduction}
\label{sec:litintro}
% Overview of satellite pose estimation challenges for Earth observation
% Importance of accurate pose determination for nanosatellites and CubeSats
% Traditional vs. vision-based approaches
% Chapter organization and scope

\mysection{Satellite Position and Attitude Determination Systems}{Satellite Position and Attitude Determination Systems}
\label{sec:litposatt}

\mysubsection{Position Determination Methods}{Position Determination Methods}
\label{sec:posdet}
% Global Navigation Satellite Systems (GNSS/GPS)
% - Accuracy limitations and availability
% - Power consumption considerations for CubeSats


% Ground-Based Tracking Systems
% - Two-Line Element (TLE) sets and orbital propagation
% - Ground station ranging and limitations
% Relative positioning techniques

\mysubsection{Attitude Determination Systems}{Attitude Determination Systems}
\label{sec:attdet}
% Star Trackers
% - High accuracy (arc-second level) but high cost and power
% - Current CubeSat star tracker solutions (ST200, ST400)
% Coarse Sun Sensors
% - Low accuracy but reliable and low power
% - Typical accuracy ranges and applications
% Magnetometers
% - Three-axis magnetometers for attitude estimation
% - TRIAD and QUEST algorithms
% Gyroscopes and IMUs
% - MEMS gyroscopes in CubeSat applications
% - Drift characteristics and calibration requirements
% Multi-sensor fusion approaches
% - Sensor complementarity and redundancy
% - Cost-accuracy trade-offs for small satellites

\mysection{Earth Observation Satellite Systems and Imaging Technologies}{Earth Observation Satellite Systems and Imaging Technologies}
\label{sec:rmtsnsing}

\mysubsection{Heritage Earth Observation Missions}{Heritage Earth Observation Missions}
\label{sec:heritage}
% Landsat Series
% - Imaging capabilities and geometric accuracy requirements
% - Camera types: pushbroom vs. whiskbroom scanners
% Sentinel Series
% - Multi-Spectral Instrument (MSI) characteristics
% - Pointing accuracy and stability requirements

\mysubsection{Commercial Earth Observation Satellites}{Commercial Earth Observation Satellites}
\label{sec:commercial}
% Planet Labs Constellation
% - PlanetScope Dove satellites: frame cameras, 3.7m resolution
% - SkySat: Cassegrain telescopes with CMOS detectors, 50cm resolution
% - RapidEye: pushbroom scanners (retired 2020)
% Maxar Technologies
% - WorldView Series: high-resolution pushbroom imaging
% - WorldView Legion: next-generation constellation, 30cm resolution
% - Camera technologies and pointing requirements
% Airbus Defence and Space
% - Pléiades Constellation: 0.5m resolution, agile imaging
% - SPOT Series: systematic Earth coverage
% NASA Earth Science Missions
% - MODIS (Terra/Aqua): 36-band imaging, 250m-1km resolution
% - VIIRS: day/night imaging capabilities
% - Specialized atmospheric and climate sensors

\mysubsection{Camera Technologies in Earth Observation}{Camera Technologies in Earth Observation}
\label{sec:cameras}
% Imaging sensor types
% - CCD vs. CMOS technologies in space applications
% - Frame cameras vs. pushbroom/whiskbroom scanners
% Pointing accuracy requirements by resolution
% - Sub-meter imagery: arc-second level pointing
% - Moderate resolution: arc-minute level acceptable
% Calibration and stability requirements

\mysubsection{Emerging Satellite Constellations}{Emerging Satellite Constellations}
\label{sec:emerging}
% Next-generation commercial systems
% - Planet Pelican: 30cm resolution successor to SkySat
% - Umbra: SAR constellation for all-weather imaging
% - Satellogic: multispectral and hyperspectral capabilities
% - Tanager-1: hyperspectral Earth observation

\mysection{Computer Vision for Satellite Applications}{Computer Vision for Satellite Applications}
\label{sec:featuredetection}

\mysubsection{Classical Feature Detection Methods}{Classical Feature Detection Methods}
\label{sec:classical}
% Scale-Invariant Feature Transform (SIFT)
% - 128-dimensional descriptors and invariance properties
% - Performance with satellite imagery and multi-temporal matching
% - Computational requirements for space applications
% Speeded-Up Robust Features (SURF)
% - Computational advantages over SIFT
% - 64-dimensional descriptors and performance trade-offs
% Oriented FAST and Rotated BRIEF (ORB)
% - Real-time performance capabilities
% - Suitability for resource-constrained platforms
% Comparative analysis for satellite pose estimation
% - Robustness to lighting changes and atmospheric effects
% - Feature matching accuracy under geometric distortions

\mysubsection{Earth Feature Tracking and Landmark Recognition}{Earth Feature Tracking and Landmark Recognition}
\label{sec:tracking}
% Ground feature selection criteria
% - Persistent landmarks vs. dynamic features
% - Urban vs. natural feature reliability
% Multi-temporal feature tracking
% - Seasonal and lighting variations
% - Long-term feature stability

\mysection{Vision-Based Pose Estimation Techniques}{Vision-Based Pose Estimation Techniques}
\label{sec:visionpose}

\mysubsection{Camera-Based Navigation Systems}{Camera-Based Navigation Systems}
\label{sec:SLAM}
% Simultaneous Localization and Mapping (SLAM)
% - Visual SLAM approaches for satellite applications
% - Landmark initialization and management
% Visual odometry techniques
% - Monocular vs. stereo approaches
% - Integration with orbital dynamics

\mysubsection{Geometric Pose Estimation Methods}{Geometric Pose Estimation Methods}
\label{sec:geometric}
% Perspective-n-Point (PnP) solutions
% - 2D-3D correspondence establishment
% - Robust estimation techniques (RANSAC)
% Feature-based pose recovery
% - Essential matrix and fundamental matrix estimation
% - Multi-view geometry applications

\mysection{State Estimation and Sensor Fusion}{State Estimation and Sensor Fusion}
\label{sec:stateestimation}

\mysubsection{Filtering Techniques for Satellite Applications}{Filtering Techniques for Satellite Applications}
\label{sec:filtering}
% Extended Kalman Filter (EKF)
% - Nonlinear state estimation for attitude and position
% - Quaternion-based attitude representations
% Unscented Kalman Filter (UKF)
% - Handling nonlinear measurement models
% - Computational considerations for CubeSats
% Particle Filters
% - Multi-modal estimation and robustness
% - Computational complexity trade-offs

\mysubsection{Multi-Sensor Fusion Architectures}{Multi-Sensor Fusion Architectures}
\label{sec:fusion}
% Vision-IMU integration
% - Tightly coupled vs. loosely coupled approaches
% - Error state estimation techniques
% Vision-GNSS fusion
% - Complementary sensor characteristics
% - Relative vs. absolute positioning

\mysubsection{Robustness and Reliability Techniques}{Robustness and Reliability Techniques}
\label{sec:robustness}
% Outlier detection and rejection
% - Statistical methods for measurement validation
% - Robust estimation techniques
% Failure detection and accommodation
% - Sensor redundancy and graceful degradation
% - Performance monitoring and health assessment

\mysection{Earth-Tracking Systems for Satellite Pose Estimation}{Earth-Tracking Systems for Satellite Pose Estimation}
\label{sec:earthtracking}

\mysubsection{Ground Feature Databases and Maps}{Ground Feature Databases and Maps}
\label{sec:databases}
% Global elevation models and terrain databases
% - SRTM, ASTER GDEM availability and accuracy
% Landmark databases for navigation
% - Persistent feature catalogs
% - Real-time vs. pre-computed approaches

\mysubsection{Applications and Performance Requirements}{Applications and Performance Requirements}
\label{sec:applications}
% Earth observation pointing requirements
% - Target acquisition and tracking accuracy
% - Image quality and geometric fidelity
% Autonomous navigation capabilities
% - Reduced ground contact and operational autonomy

\mysection{Literature Gap Analysis and Research Opportunities}{Literature Gap Analysis and Research Opportunities}
\label{sec:gaps}
% Current limitations in CubeSat pose estimation
% - Cost vs. accuracy trade-offs
% - Limited availability of space-qualified high-accuracy sensors
% Vision-based solutions for small satellites
% - Leveraging existing imaging payloads
% - Real-time processing constraints and solutions
% Integration challenges and opportunities
% - Multi-sensor fusion complexity
% - Ground truth validation difficulties

\mysection{Conclusion}{Conclusion}
\label{sec:litconclusion}
% Summary of key findings from literature review
% Identification of research gaps and opportunities
% Justification for proposed vision-based approach
% Preview of methodology and contributions
