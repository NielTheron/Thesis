\mychapter{Introduction}{Introduction}{}
\label{chap:introduction}

\mysection{Problem Background}{Problem Background}
\label{sec:overview}

- Satellites are getting smaller
- Because this leads to stallites having reduced costs and timelines
- This is enables by the miniraturisation of electronics

- One of the big industires in satellites is remote sensing
- Remote Sening is the application where satellites are used to monitor the Earth
- One of the applications is to take images of the Earth

- This leads to the problem that high accuracy is needed to take images of the targets on the Earth's surface
- COTS components which is mainly used on small satellites lack the accuracy needed
- Magnetometers is to low of an accuracy
- Star Trackers have the right accuracy, but is expensive

\mysection{Proposed Solution}{Proposed Solution}
\label{sec:description}

- Proposed solution is to develop an estimation algorithm that can estimate the full state of the satellite
- The Full State of a Satellite is its postition in Space and its attitude or its orientation in space.
- The satellite uses the imager itself to determine position and attitude.
- This can lead to reduce costs as the satellite is using an instrument which is already onboard the satellite.
- Utilising the components when it is idle
- Observing the target directly


\mysection{Document Outline}{Document Outline}
\label{sec:outline}

- Chapter 2: Wil investigate previous sensors that is being used to determine Propose
- Previous techniques estimating the pose
- Some light touching on feature detection as this is crucial to the pose estimation system

- Chapter 3: Wil introduce the modelling of the system
- Rigid Body Kinematics
- Position Kinematics
- Attitude Kinematics
- Kalman Filters
- Extended Kalman Filters

- Chapter 4: Measurement Generation
- Feature detection
- PinHole Camera Model.
- The Plant
- The Plant Model
- The Measurment Model

- Chapter 5: State estimation
- The Extended Kalman Filter
- Update Step
- Prediction Step
- Simulator

- Chapter 6 is results

- Chapter 7 is Conclusion
- Future Work