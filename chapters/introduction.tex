\mychapter{Introduction}{Introduction}{}
\label{chap:introduction}

%==============================================================================================================================================
\mysection{Problem Background}{Problem Background}
\label{sec:overview}

\noindent
In recent years, the satellite industry has undergone a significant paradigm shift toward smaller, more cost-effective platforms. This trend is largely enabled by 
the ongoing miniaturization of electronics and sensor technology, which has allowed for the development of highly capable small satellites, including nanosatellites and 
CubeSats. The primary drivers behind this shift are the substantially reduced development costs and accelerated timelines associated with these smaller platforms, which 
have made space accessible for academic, commercial, and governmental entities \cite{Villela,bouwmeester2010}.
\vspace{0.5cm}

\noindent
A predominant application within this rapidly growing market is Earth Observation (EO), which involves using satellite-borne sensors to monitor the planet for a 
wide range of scientific and commercial purposes. A key requirement for many EO missions is the ability to capture high-resolution imagery (for a GSD in the range of 5-10m) of specific, pre-defined targets 
on the Earth's surface. This, in turn, necessitates a high degree of pointing accuracy (typically in the range of 0.01$^\circ$ or better) from the satellite's Attitude Determination 
and Control System (ADCS) \cite{Toth2016, Lesikar, Zhang2024}.
\vspace{0.5cm}

\noindent
This requirement introduces the critical technical challenge at the heart of this research. The low-cost philosophy of small satellites mandates the use of 
Commercial-Off-The-Shelf (COTS) components, creating a trade-off between cost and performance \cite{Villela,bouwmeester2010}. A significant gap exists between the pointing accuracy required for meaningful 
remote sensing and what is achievable with standard, low-cost ADCS sensors. For instance, magnetometers provide robust, continious attitude information but lack the necessary 
precision, while high-performance star trackers can deliver the required accuracy but are often prohibitively expensive \cite{Ibrahim2020}. This dilemma creates a pressing need for novel, 
cost-effective solutions that can bridge this performance gap.

%================================================================================================================================================
\mysection{Proposed Solution: Visual Navigation for Pose Estimation}{Proposed Solution: Visual Navigation for Pose Estimation}
\label{sec:solution}

\noindent
To address this challenge, this thesis investigates an autonomous visual navigation approach to determine the satellite's full six-degree-of-freedom (6-DOF) pose, 
encompassing both its three-dimensional position and attitude. Instead of relying on traditional external systems like GNSS, the satellite performs a type of "reverse GPS" by 
identifying known ground features in its own imagery to compute its state. The primary novelty of this work is the application of visual navigation techniques, 
commonly used in robotics, to Earth-orbiting satellites. Traditional satellite ADCS architectures rely on a suite of dedicated sensors, whereas this 
thesis demonstrates a "payload-as-a-sensor" approach, repurposing the primary science imager as the main navigation instrument. Figure \ref{fig:satellite_concept} 
illustrates the core mechanics of this problem, showing the relationship between the satellite and the Earth in the Earth-Centered Inertial (ECI) reference frame and 
how it relates to the 6-DOF navigation challenge.

\begin{figure}[H]
    \centering
    \includegraphics[width=0.8\textwidth]{figures/stateestimation/OverallSystem.pdf}
    \caption{Conceptual diagram of the proposed visual navigation problem, illustrating the key reference frames (ECI, Body) and the geometric relationship between the 
    satellite's imager and known features on the Earth's surface.}
    \label{fig:satellite_concept}
\end{figure}

\noindent
This "payload-as-a-sensor" methodology presents several significant advantages. It offers a potential reduction in system cost and complexity by eliminating the need for 
expensive, high-fidelity sensors. It also enhances system efficiency by enabling the dual-use of the primary imager during periods when it might otherwise be idle. 
Finally, it provides a direct, relative measurement of the satellite's pointing with respect to the Earth, which is fundamentally what is required for high-accuracy EO missions.
\vspace{0.5cm}

\noindent
The core technical challenge lies in transforming raw imagery into a precise pose estimate. Although conceptually related to Simultaneous Localization and Mapping (SLAM), the problem considered here involves only the localization component. Unlike full SLAM, where both the environment and the pose are estimated simultaneously, the "map", a catalogue of georeferenced ground features, is assumed to be known a priori. The focus is therefore on the localization task, which entails addressing three key challenges:

\begin{itemize}
    \item \textbf{Geometric Inversion:} The complex inverse problem of determining a 6-DOF pose from 2D image projections of known 3D landmarks.
    \item \textbf{Uncertainty Management:} The real-time handling of measurement noise and dynamic orbital motion within a recursive filter.
    \item \textbf{Sensor Fusion Integration:} Incorporating visual measurements into the traditional Attitude Determination and Control System (ADCS) through sensor 
    fusion techniques, enabling synergistic use of camera, gyroscope, magnetometer, and other sensor data.
\end{itemize}
\vspace{0.5cm}

\noindent
To maintain a clear research focus, this thesis presumes that the feature matching problem, where the correct association of image features to catalogue entries, is solved. 
This necessary simplification allows the research to concentrate on its core contribution: developing a state estimation framework that optimally leverages the geometric 
data from these established feature correspondences.

%============================================================================================================================================
\mysection{Contributions and Thesis Outline}{Contributions and Thesis Outline}
\label{sec:contributions_outline}

\noindent
The main contributions of this thesis are:
\begin{itemize}
    \itemsep0em
    \item The formulation of a novel satellite state estimation framework that adapts visual odometry principles from robotics for Earth-orbiting applications.
    \item The development of a "payload-as-a-sensor" model that integrates the primary imager as a navigation sensor within an Extended Kalman Filter.
    \item The design and implementation of a complete simulation environment to validate the performance of the proposed visual navigation system under various operational conditions.
\end{itemize}

\noindent
The remainder of this document is structured as follows:
\vspace{0.5cm}

\noindent
\textbf{Chapter 2} presents a comprehensive literature review, examining both relative and absolute pose estimation methods. It then explores traditional pose estimation
and visual-based navigation approaches, followed by an investigation of contemporary image processing techniques, state estimation methods, and sensor fusion strategies 
involving traditional sensors.
\vspace{0.5cm}

\noindent
\textbf{Chapter 3} establishes the mathematical framework for the simulation, detailing rigid body mechanics, reference frame transformations, and the models for auxiliary sensors.
\vspace{0.5cm}

\noindent
\textbf{Chapter 4} details the development of the vision-based Earth Tracker, including the camera model, lens distortion simulation, and the algorithms for generating measurement vectors and feature catalogues.
\vspace{0.5cm}

\noindent
\textbf{Chapter 5} focuses on the design of the state estimation algorithm, presenting the theoretical background of recursive estimation and the detailed derivation of the Extended Kalman Filter (EKF) for this specific application.
\vspace{0.5cm}

\noindent
\textbf{Chapter 6} presents the simulation results and a thorough performance analysis of the system under various test conditions, including sensor noise, feature availability, and distortion effects.
\vspace{0.5cm}

\noindent
\textbf{Chapter 7} concludes the thesis by summarizing the key findings, discussing the limitations of the work, and proposing avenues for future research.
\vspace{0.5cm}