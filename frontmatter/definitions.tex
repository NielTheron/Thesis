\newglossaryentry{ADCS}
{
	name=Attitude Determination and Control System,
	text=attitude determination and control system,
	description={A subsystem responsible for determining and controlling the orientation of a satellite in space. It uses sensors such as star trackers, magnetometers, and sun sensors to measure attitude, and actuators such as reaction wheels or magnetorquers to adjust it.},
	plural=attitude determination and control systems
}

\newglossaryentry{Attitude}
{ 
	name=Attitude,
	text=attitude,
	description={The orientation of a satellite in space, typically described by the rotational relationship between the body-fixed coordinate system and a reference coordinate system.},
	plural=attitudes
}

\newglossaryentry{Back-projection}
{
	name=Back-projection,
	text=back-projection,
	description={The process of converting a 2D feature point in an image back into a 3D feature vector in space, effectively reversing the camera projection.}
}

\newglossaryentry{Batch EKF}
{
	name=Batch EKF,
	text=batch EKF,
	description={An Extended Kalman Filter architecture where all sensor measurements are combined into a single complex measurement model and processed concurrently.}
}

\newglossaryentry{Body Reference Frame}
{
	name=Body Reference Frame,
	text=body reference frame,
	description={A coordinate system fixed to the satellite body, with origin at the center of mass. The axes are typically aligned with the satellite's principal axes of inertia.},
	plural=body reference frames
}

\newglossaryentry{Camera Reference Frame}
{
	name=Camera Reference Frame,
	text=camera reference frame,
	description={A coordinate system attached to the camera, with the z-axis pointing along the optical axis and the origin at the optical center.},
	plural=camera reference frames
}

\newglossaryentry{Catalogue}
{
	name=Catalogue,
	text=catalogue,
	description={A database of geolocated features with known coordinates used for matching and pose estimation. In this work, it contains latitude, longitude, and altitude of detected features.},
	plural=catalogues
}

\newglossaryentry{Chromatic Aberration}
{
	name=Chromatic Aberration,
	text=chromatic aberration,
	description={A type of lens distortion where different wavelengths of light focus at different distances from the lens, causing color fringing at high-contrast edges.}
}

\newglossaryentry{COTS}
{
	name=Commercial Off-The-Shelf,
	text=commercial off-the-shelf,
	description={Components that are readily available for purchase from commercial suppliers, as opposed to custom-designed or space-rated hardware. COTS components reduce cost but may sacrifice performance or reliability.}
}

\newglossaryentry{CubeSat}
{
	name=CubeSat,
	text=CubeSat,
	description={A type of miniaturized satellite made up of multiples of 10×10×10 cm cubic units, designed using standardized specifications to reduce cost and development time.},
	plural=CubeSats
}

\newglossaryentry{DCM}
{
	name=Direction Cosine Matrix,
	text=direction cosine matrix,
	description={A 3×3 rotation matrix that describes the orientation of one coordinate frame with respect to another. Also called a rotation matrix.},
	plural=direction cosine matrices
}

\newglossaryentry{DEM}
{
	name=Digital Elevation Model,
	text=digital elevation model,
	description={A 3D representation of terrain surface that incorporates detailed topographic information including mountains, valleys, and other local features at high spatial resolution.},
	plural=digital elevation models
}

\newglossaryentry{Dzhanibekov Effect}
{
	name=Dzhanibekov Effect,
	text=Dzhanibekov effect,
	description={An instability phenomenon where rotation about the intermediate principal axis of inertia causes periodic flipping of the body's orientation.}
}

\newglossaryentry{Earth-Centered Earth-Fixed}
{
	name=Earth-Centered Earth-Fixed,
	text=Earth-centered Earth-fixed,
	description={A Cartesian coordinate system with origin at Earth's center of mass, rotating with the Earth. The z-axis points toward the North Pole, and the x-axis intersects the Prime Meridian at the Equator.}
}

\newglossaryentry{Earth-Centered Inertial}
{
	name=Earth-Centered Inertial,
	text=Earth-centered inertial,
	description={A non-rotating coordinate system with origin at Earth's center of mass. The z-axis points toward the North Pole, and the x-axis points toward the Vernal Equinox. Used for orbital dynamics and space navigation.}
}

\newglossaryentry{Earth Observation}
{
	name=Earth Observation,
	text=Earth observation,
	description={The use of satellite-borne sensors to monitor the planet for scientific and commercial purposes, typically involving high-resolution imagery of specific targets on Earth's surface.}
}

\newglossaryentry{Earth Tracker}
{
	name=Earth Tracker,
	text=Earth Tracker,
	description={A vision-based sensor developed in this work that uses satellite imagery and feature detection to estimate the satellite's pose by matching observed features to a geolocated catalogue.}
}

\newglossaryentry{EKF}
{
	name=Extended Kalman Filter,
	text=extended Kalman filter,
	description={A nonlinear variant of the Kalman filter that linearizes nonlinear functions using first-order Taylor expansion around the current state estimate.}
}

\newglossaryentry{Euler Angles}
{
	name=Euler Angles,
	text=Euler angles,
	description={A set of three angles (typically roll, pitch, and yaw) used to describe the orientation of a rigid body. Susceptible to gimbal lock singularities.},
	plural=Euler angles
}

\newglossaryentry{Extrinsic Parameters}
{
	name=Extrinsic Parameters,
	text=extrinsic parameters,
	description={Camera parameters that define the position and orientation of the camera reference frame relative to another reference frame, typically expressed as a rotation and translation.},
	plural=extrinsic parameters
}

\newglossaryentry{Feature Detection}
{
	name=Feature Detection,
	text=feature detection,
	description={The process of identifying distinctive and repeatable points within an image that can be used for image matching, tracking, and pose estimation.}
}

\newglossaryentry{Feature Vector}
{
	name=Feature Vector,
	text=feature vector,
	description={A 3D vector representation of a feature point in space, obtained through back-projection from a 2D image pixel.},
	plural=feature vectors
}

\newglossaryentry{Field of View}
{
	name=Field of View,
	text=field of view,
	description={The angular extent of the observable area visible through a camera or sensor at any given moment, typically measured as the angle between opposite image boundaries.}
}

\newglossaryentry{Geoid}
{
	name=Geoid,
	text=geoid,
	description={An equipotential surface of Earth's gravity field that corresponds to mean sea level, accounting for gravitational anomalies but not local terrain variations.}
}

\newglossaryentry{Geolocation}
{
	name=Geolocation,
	text=geolocation,
	description={The process of determining the geographic coordinates (latitude, longitude, altitude) of a feature or object.}
}

\newglossaryentry{Gimbal Lock}
{
	name=Gimbal Lock,
	text=gimbal lock,
	description={A mathematical singularity that occurs when using Euler angles, where two rotational axes become aligned and one degree of freedom is lost.}
}

\newglossaryentry{GNSS}
{
	name=Global Navigation Satellite System,
	text=global navigation satellite system,
	description={A satellite-based system that provides autonomous geospatial positioning. Examples include GPS, GLONASS, Galileo, and BeiDou.}
}

\newglossaryentry{Ground Sampling Distance}
{
	name=Ground Sampling Distance,
	text=ground sampling distance,
	description={The physical distance between the centers of two adjacent pixels measured on the ground in satellite imagery. Smaller GSD indicates higher spatial resolution.}
}

\newglossaryentry{Homogeneous Transformation}
{
	name=Homogeneous Transformation,
	text=homogeneous transformation,
	description={A 4×4 matrix that combines rotation and translation into a single transformation, using homogeneous coordinates where position vectors are augmented with a fourth element.},
	plural=homogeneous transformations
}

\newglossaryentry{IGRF}
{
	name=International Geomagnetic Reference Field,
	text=international geomagnetic reference field,
	description={A standard mathematical representation of Earth's main magnetic field and its secular variation, used for navigation and attitude determination.}
}

\newglossaryentry{Innovation}
{
	name=Innovation,
	text=innovation,
	description={In Kalman filtering, the difference between an actual sensor measurement and the predicted measurement, representing new information available to the filter.},
	plural=innovations
}

\newglossaryentry{Intrinsic Parameters}
{
	name=Intrinsic Parameters,
	text=intrinsic parameters,
	description={Camera parameters internal to the camera that define how 3D points are projected onto the 2D image plane, including focal length, pixel size, and principal point offset.},
	plural=intrinsic parameters
}

\newglossaryentry{J2 Perturbation}
{
	name=J2 Perturbation,
	text=J2 perturbation,
	description={An orbital perturbation caused by Earth's oblateness (flattening at the poles), represented by the second zonal harmonic coefficient in the gravitational potential.}
}

\newglossaryentry{J2000}
{
	name=J2000,
	text=J2000,
	description={A standard epoch (January 1, 2000, 12:00 TT) used as a reference for celestial coordinate systems and orbital elements.}
}

\newglossaryentry{Jacobian}
{
	name=Jacobian,
	text=Jacobian,
	description={A matrix of partial derivatives representing the local linear approximation of a nonlinear function, used in the EKF to linearize process and measurement models.},
	plural=Jacobians
}

\newglossaryentry{Kalman Gain}
{
	name=Kalman Gain,
	text=Kalman gain,
	description={A matrix that determines the weight given to the innovation (measurement residual) when updating the state estimate in a Kalman filter.}
}

\newglossaryentry{LEO}
{
	name=Low Earth Orbit,
	text=low Earth orbit,
	description={An orbital regime typically between 160 and 2000 km altitude, characterized by relatively short orbital periods and higher atmospheric drag than higher orbits.}
}

\newglossaryentry{LVLH}
{
	name=Local Vertical Local Horizon,
	text=local vertical local horizon,
	description={An orbital reference frame that moves with the satellite, with the z-axis pointing toward Earth's center (nadir), the y-axis perpendicular to the orbital plane, and the x-axis tangent to the orbit.}
}

\newglossaryentry{Measurement Model}
{
	name=Measurement Model,
	text=measurement model,
	description={A mathematical function that relates the system states to sensor observations, used in state estimation to predict what measurements should be observed given a state estimate.},
	plural=measurement models
}

\newglossaryentry{Motion Model}
{
	name=Motion Model,
	text=motion model,
	description={A mathematical model describing how the system states evolve over time, incorporating dynamics, kinematics, and external forces.},
	plural=motion models
}

\newglossaryentry{Multispectral Imager}
{
	name=Multispectral Imager,
	text=multispectral imager,
	description={A sensor that captures images in multiple specific wavelength bands, including visible and non-visible portions of the electromagnetic spectrum.},
	plural=multispectral imagers
}

\newglossaryentry{Nadir}
{
	name=Nadir,
	text=nadir,
	description={The direction pointing directly toward Earth's center from a satellite's position. A nadir-pointing camera looks straight down at the surface.}
}

\newglossaryentry{Nanosatellite}
{
	name=Nanosatellite,
	text=nanosatellite,
	description={A satellite with mass between 1 and 10 kg, typically built using standardized CubeSat specifications.},
	plural=nanosatellites
}

\newglossaryentry{Off-Nadir Angle}
{
	name=Off-Nadir Angle,
	text=off-nadir angle,
	description={The angle between the nadir direction and the camera's pointing direction. Increases geometric distortion and affects ground sampling distance.}
}

\newglossaryentry{ORB}
{
	name=Oriented FAST and Rotated BRIEF,
	text=oriented FAST and rotated BRIEF,
	description={A fast binary feature descriptor that provides scale and rotation invariance with lower computational cost than SIFT or SURF.}
}

\newglossaryentry{Payload-as-a-Sensor}
{
	name=Payload-as-a-Sensor,
	text=payload-as-a-sensor,
	description={An approach where the primary science instrument (typically a camera) is repurposed as a navigation sensor to reduce system cost and complexity.}
}

\newglossaryentry{Pinhole Camera Model}
{
	name=Pinhole Camera Model,
	text=pinhole camera model,
	description={An idealized geometric camera model where light rays pass through a single point (the optical center) to form an inverted image on an image plane.}
}

\newglossaryentry{Pixel Pitch}
{
	name=Pixel Pitch,
	text=pixel pitch,
	description={The physical size of a single pixel on an imaging sensor, typically measured in micrometers.}
}

\newglossaryentry{PnP}
{
	name=Perspective-n-Point,
	text=perspective-n-point,
	description={A problem in computer vision of estimating the pose of a camera given 3D points in space and their corresponding 2D projections in an image.}
}

\newglossaryentry{Pose}
{
	name=Pose,
	text=pose,
	description={The combination of a satellite's position and attitude, representing its complete six-degree-of-freedom state in space.},
	plural=poses
}

\newglossaryentry{Pushbroom Sensor}
{
	name=Pushbroom Sensor,
	text=pushbroom sensor,
	description={An imaging sensor that captures one row of pixels at a time, relying on the satellite's forward motion to build up a complete 2D image.},
	plural=pushbroom sensors
}

\newglossaryentry{Quaternion}
{
	name=Quaternion,
	text=quaternion,
	description={A four-element mathematical representation of rotation consisting of one scalar and three vector components, avoiding gimbal lock singularities inherent in Euler angles.},
	plural=quaternions
}

\newglossaryentry{Radial Distortion}
{
	name=Radial Distortion,
	text=radial distortion,
	description={A lens aberration where magnification varies with radial distance from the optical center, causing straight lines to appear curved (barrel or pincushion distortion).}
}

\newglossaryentry{RANSAC}
{
	name=Random Sample Consensus,
	text=random sample consensus,
	description={A robust algorithm for estimating model parameters in the presence of outliers by iteratively fitting models to random subsets of data.}
}

\newglossaryentry{Raycasting}
{
	name=Raycasting,
	text=raycasting,
	description={A technique for determining the intersection of a ray with a 3D surface, used in this work to geolocate features by finding where feature vectors intersect Earth's surface.}
}

\newglossaryentry{Recursive Estimation}
{
	name=Recursive Estimation,
	text=recursive estimation,
	description={A method of updating state estimates by combining prior state distributions with new measurements at each time step, rather than reprocessing all historical data.}
}

\newglossaryentry{Reference Frame}
{
	name=Reference Frame,
	text=reference frame,
	description={A coordinate system with a defined origin and orientation used to describe positions, velocities, and rotations in space.},
	plural=reference frames
}

\newglossaryentry{Rigid Body Dynamics}
{
	name=Rigid Body Dynamics,
	text=rigid body dynamics,
	description={The study of the motion of solid objects that do not deform under applied forces, governed by Newton-Euler equations.}
}

\newglossaryentry{Sensor Fusion}
{
	name=Sensor Fusion,
	text=sensor fusion,
	description={The process of combining measurements from multiple sensors to produce more accurate and reliable state estimates than could be achieved using any single sensor.}
}

\newglossaryentry{Sequential Update}
{
	name=Sequential Update,
	text=sequential update,
	description={An EKF architecture where sensor measurements are processed one at a time rather than simultaneously, allowing for asynchronous sensor integration.}
}

\newglossaryentry{SIFT}
{
	name=Scale-Invariant Feature Transform,
	text=scale-invariant feature transform,
	description={A feature detection algorithm that identifies distinctive keypoints invariant to scale, rotation, and illumination changes, widely used in image matching and recognition.}
}

\newglossaryentry{SLAM}
{
	name=Simultaneous Localization and Mapping,
	text=simultaneous localization and mapping,
	description={A technique where a mobile system simultaneously builds a map of its environment while determining its location within that map.}
}

\newglossaryentry{Snapshot Sensor}
{
	name=Snapshot Sensor,
	text=snapshot sensor,
	description={An imaging sensor with a global shutter that captures an entire 2D array of pixels simultaneously in a single exposure.},
	plural=snapshot sensors
}

\newglossaryentry{Spatial Resolution}
{
	name=Spatial Resolution,
	text=spatial resolution,
	description={The level of detail visible in an image, typically measured by ground sampling distance. Higher spatial resolution allows finer features to be distinguished.}
}

\newglossaryentry{State Estimation}
{
	name=State Estimation,
	text=state estimation,
	description={The process of determining the values of system states (position, velocity, attitude, etc.) using mathematical models and sensor measurements.}
}

\newglossaryentry{State Vector}
{
	name=State Vector,
	text=state vector,
	description={A vector containing all the variables that describe the complete state of a system at a given time.},
	plural=state vectors
}

\newglossaryentry{Sun-Synchronous Orbit}
{
	name=Sun-Synchronous Orbit,
	text=sun-synchronous orbit,
	description={A near-polar orbit where the satellite passes over any given point at approximately the same local solar time, useful for consistent lighting conditions in Earth observation.},
	plural=sun-synchronous orbits
}

\newglossaryentry{SURF}
{
	name=Speeded-Up Robust Features,
	text=speeded-up robust features,
	description={A feature detection algorithm that approximates SIFT using integral images for faster computation while maintaining robustness to scale and rotation.}
}

\newglossaryentry{SWAP-C}
{
	name=SWaP-C,
	text=SWaP-C,
	description={Size, Weight, Power, and Cost constraints that drive the design philosophy of small satellites and CubeSats.}
}

\newglossaryentry{Swath}
{
	name=Swath,
	text=swath,
	description={The total ground area covered by a satellite imager in a single pass, determined by the field of view and orbital altitude.}
}

\newglossaryentry{Tangential Distortion}
{
	name=Tangential Distortion,
	text=tangential distortion,
	description={A lens distortion that occurs when the lens and image sensor are not perfectly parallel, causing the image to appear tilted or skewed.}
}

\newglossaryentry{Temporal Resolution}
{
	name=Temporal Resolution,
	text=temporal resolution,
	description={The frequency with which a satellite revisits a site or target area, determining how often updated imagery is available.}
}

\newglossaryentry{TLE}
{
	name=Two-Line Elements,
	text=two-line elements,
	description={A standardized data format containing the six classical Keplerian orbital elements used to describe satellite orbits, distributed by NORAD.}
}

\newglossaryentry{Two-Body Problem}
{
	name=Two-Body Problem,
	text=two-body problem,
	description={A simplified orbital mechanics problem considering only the gravitational attraction between two bodies, ignoring perturbations from other forces.}
}

\newglossaryentry{Visual Odometry}
{
	name=Visual Odometry,
	text=visual odometry,
	description={The process of determining the position and orientation of a vehicle by analyzing sequences of images from onboard cameras.}
}

\newglossaryentry{WGS84}
{
	name=World Geodetic System 1984,
	text=World Geodetic System 1984,
	description={A standard geodetic reference system that models Earth as an oblate spheroid with specified semi-major and semi-minor axes, widely used in GPS and navigation.}
}

\newglossaryentry{Whiskbroom Sensor}
{
	name=Whiskbroom Sensor,
	text=whiskbroom sensor,
	description={An imaging sensor that captures one pixel at a time by scanning across the swath using a rotating mirror, relying on satellite motion to build a complete image.},
	plural=whiskbroom sensors
}