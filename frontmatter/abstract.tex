\chapter*{Abstract}
\addcontentsline{toc}{chapter}{Abstract}
\makeatletter\@mkboth{}{\scshape Abstract}\makeatother

\noindent Pose estimation on nanosatellites is still an on going topic of interest.
It is important for satellite to know there position and attitude to do accurate target tracking.
Traditional solutions to the pose estimation problem is mainly star trackers, which looks at the constalations of stars to determine the attitude
and GPS to determine the position of the satellite along with other sensors like magnetometers and coarse sun sensors.

\vspace{0.5cm}

\noindent In this thesis, a sensor is developed that utilises the onboard satellite imager, to estimate the position and the attitude of the satellite.
The sensor uses a camera model to take pictures of the Earth surface, a feature detector is ran on the image using scale invariant feature transform (SIFT)
to identify and establish corrospondance of features. A full state kinematic estimator using the extended Kalman Filter (EKF) based on the simultanous
localisation and mapping (SLAM) approach. The filter makes used of feature vectors and feature discripters detected on the image. This is used to estimate
attitude and position of the satellite.

\vspace{0.5cm}

\noindent An simulation environment in MATLAB is developed to propagate a satellite and determine the ground truth pose. Several traditional sensors
like the star tracker and magnetometer and GPS to be able to compare the Earth Tracker and create the possiblity to fuse the sensors and determine the accuracy.
Results show that the filter estimates the system states successfully. It is concluded that \dots


\newpage
\selectlanguage{afrikaans}

\chapter*{Uittreksel}
\addcontentsline{toc}{chapter}{Uittreksel}
\makeatletter\@mkboth{}{\scshape Uittreksel}\makeatother

\selectlanguage{english}